\documentclass[a4paper]{article}
 
\usepackage[margin=1in]{geometry}
\usepackage{amsmath,amsthm,amssymb,bbm,wasysym}
\usepackage[czech]{babel}
\usepackage[utf8]{inputenc}
\usepackage[T1]{fontenc}
\usepackage{tikz}
\usepackage{graphicx}
\usepackage{enumitem}
\usepackage{tabto}
\usepackage{amsmath}

\graphicspath{ {./} }

\DeclareMathOperator{\Ex}{\mathbb{E}} % střední hodnota X pomocí $\Ex X$

\newcommand{\N}{\mathbb{N}} % přirozená čísla
\newcommand{\Z}{\mathbb{Z}} % celá čísla
\newcommand{\R}{\mathbb{R}} % reálná čísla

\renewcommand{\qed}{\hfill\blacksquare} % Quod Est Demonstratum (QED) 

% tohle je pro prostředí úkolů
\newenvironment{ukol}[3][]{\begin{trivlist} 
\item[\hskip \labelsep {\bfseries #1}\hskip \labelsep {\bfseries #2}]}{\end{trivlist}}

\linespread{1.15}
 
\begin{document}
 
% --------------------------------------------------------------
%                         Začni ZDE
% --------------------------------------------------------------
 
\title{ Evoluční algoritmy 1 \\ 3. domácí úkol
        } 
\author{Martin Gráf}
\date{17.11.2023}

\maketitle

Úkolem bylo implementovat různbé postupy řešení spojité optimalizace pomocí evolučních algoritmů.

\begin{ukol}{Klasické a adaptivní}

V jednoduchosti je síla. Ozkoušeli jsme různé parametry i kombinace operátorů, ale jako zdaleka nejúspěšnější se ukázala adaptivní mutace. Adaptivní 
mutace přitom pouze postupně zmenšovala velikost změn mutací, dosáhla tak ovšem v některých případech jako jedinná optima. Jako další stojí za zmínku vážený
průměr rodičů při křížení s randomizovanou váhou, což ale kupodivu výkon takřka plošně zhoršilo.

\begin{center}
	\begin{tabular}{ c }
		\includegraphics[width=1\linewidth]{./def.f01} \\ 
		\includegraphics[width=1\linewidth]{./def.f02} \\ 
	\end{tabular}
\end{center}

\begin{center}
	\begin{tabular}{ c }
		\includegraphics[width=1\linewidth]{./def.f06} \\
		\includegraphics[width=1\linewidth]{./def.f08} \\ 
	\end{tabular}
\end{center}

\begin{center}
	\begin{tabular}{ c }
		\includegraphics[width=1\linewidth]{./def.f10} \\ 
	\end{tabular}
\end{center}

\end{ukol}

\begin{ukol}{Diferenciální operátory}

	Zde jsme bohužel dosáhli pouze kýžených výsledků. Nakonec sice fungovaly rozumně silně, za žádných podmínek se ovšem nedokázali vyrovnat ostatním přístupům.

	Adaptivní přístupy se zdá nemají obdobný efekt jako u klasických operátorů. V obou případech se prokázaly jako méně konzistentní než jejich 
	statické protějšky. Mnohem větší dopad na kvalitu výsledků měl počet párů ze kterých se jedinec tvoří.

	\begin{center}
		\begin{tabular}{ c }
			\includegraphics[width=1\linewidth]{./diff.f01} \\ 
			\includegraphics[width=1\linewidth]{./diff.f02} \\ 
		\end{tabular}
	\end{center}
	
	\begin{center}
		\begin{tabular}{ c }
			\includegraphics[width=1\linewidth]{./diff.f06} \\
			\includegraphics[width=1\linewidth]{./diff.f08} \\ 
		\end{tabular}
	\end{center}
	
	\begin{center}
		\begin{tabular}{ c }
			\includegraphics[width=1\linewidth]{./diff.f10} \\ 
		\end{tabular}
	\end{center}
\end{ukol}

\begin{ukol}{Lamarckismus a Baldwinismus}

	Přes jisté problémy s interpretací popisu algoritmů jsme nakonec dosáhli překvapivě dobrých výsledků až na pár nečekaných vyjímek. Nutno podotknout,
	že Lamarckismus fungoval značně lépe než Baldwinismus. Dokonce ani kombinace obou přístupů nepomohla.

	Ve všech případech se nejlepšího řešení alespoň částečně účastnil Lamarckismus. Pravděpodobně to bude faktem, že Baldwinismus hodnotí jedince pomocí 
	fitness kterou vlastně nemají a mít ani nebudou. V některých případech značně pomáhá, že odhad fitness se shoduje s mutací která jedince čeká.

\begin{center}
	\begin{tabular}{ c }
		\includegraphics[width=1\linewidth]{./lb.f01} \\ 
		\includegraphics[width=1\linewidth]{./lb.f02} \\ 
	\end{tabular}
\end{center}

\begin{center}
	\begin{tabular}{ c }
		\includegraphics[width=1\linewidth]{./lb.f06} \\
		\includegraphics[width=1\linewidth]{./lb.f08} \\ 
	\end{tabular}
\end{center}

\begin{center}
	\begin{tabular}{ c }
		\includegraphics[width=1\linewidth]{./lb.f10} \\ 
	\end{tabular}
\end{center}
\end{ukol}

\end{document}
