\documentclass[a4paper]{article}
 
\usepackage[margin=1in]{geometry}
\usepackage{amsmath,amsthm,amssymb,bbm,wasysym}
\usepackage[czech]{babel}
\usepackage[utf8]{inputenc}
\usepackage[T1]{fontenc}
\usepackage{tikz}
\usepackage{graphicx}
\usepackage{enumitem}
\usepackage{tabto}
\usepackage{amsmath}

\graphicspath{ {./} }

\DeclareMathOperator{\Ex}{\mathbb{E}} % střední hodnota X pomocí $\Ex X$

\newcommand{\N}{\mathbb{N}} % přirozená čísla
\newcommand{\Z}{\mathbb{Z}} % celá čísla
\newcommand{\R}{\mathbb{R}} % reálná čísla

\renewcommand{\qed}{\hfill\blacksquare} % Quod Est Demonstratum (QED) 

% tohle je pro prostředí úkolů
\newenvironment{ukol}[3][]{\begin{trivlist} 
\item[\hskip \labelsep {\bfseries #1}\hskip \labelsep {\bfseries #2}]}{\end{trivlist}}

\linespread{1.15}
 
\begin{document}
 
% --------------------------------------------------------------
%                         Začni ZDE
% --------------------------------------------------------------
 
\title{ Evoluční algoritmy 1 \\ 4. domácí úkol
        } 
\author{Martin Gráf}
\date{9.12.2023}

\maketitle

Úkolem bylo implementovat různbé postupy řešení optimalizace dle několika metrik.

\begin{ukol}{Experimenty}

V jednoduchosti je síla. Ozkoušeli jsme různé parametry i kombinace operátorů, ale jako zdaleka nejúspěšnější se ukázala vážená mutace. Výchozí parametry bylo překvapivě těžké překonat, a postupy jako diferenciální mutace nebo adaptivní změna velikosti kroků zkrátka nedokázaly držet krok. Důvodem bude pravděpodobně odlišnost různých metrik. Konvergence jedince k řešení všech cílů zkrátka není tak "lineární" jako u jediného cíle, což nejspíš uvězní algoritmus v lokálních optimech (viz jejich křivky) při zmenšujících-se krocích nebo rozdílech jedinců. 

Porovnání druhotných kritérií už není tak jednoznačné. Pro každou funkci se jako nejlepší ukázal jiný přístup, i zde se ale naše vážená mutace projevila jako lepší.

\begin{center}
	\begin{tabular}{ c }
		\includegraphics[width=1\linewidth]{./ZDT1}
	\end{tabular}
	\begin{tabular}{ c }
		\includegraphics[width=1\linewidth]{./ZDT2}
	\end{tabular} 
	\begin{tabular}{ c }
		\includegraphics[width=1\linewidth]{./ZDT3}
	\end{tabular} 
	\begin{tabular}{ c }
		\includegraphics[width=1\linewidth]{./ZDT4}
	\end{tabular}
	\begin{tabular}{ c }
		\includegraphics[width=1\linewidth]{./ZDT6}
	\end{tabular}
\end{center}

\end{ukol}

\end{document}
