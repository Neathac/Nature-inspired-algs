\documentclass[a4paper]{article}
 
\usepackage[margin=1in]{geometry}
\usepackage{amsmath,amsthm,amssymb,bbm,wasysym}
\usepackage[czech]{babel}
\usepackage[utf8]{inputenc}
\usepackage[T1]{fontenc}
\usepackage{tikz}
\usepackage{graphicx}
\usepackage{enumitem}
\usepackage{tabto}
\usepackage{amsmath}

\graphicspath{ {./} }

\DeclareMathOperator{\Ex}{\mathbb{E}} % střední hodnota X pomocí $\Ex X$

\newcommand{\N}{\mathbb{N}} % přirozená čísla
\newcommand{\Z}{\mathbb{Z}} % celá čísla
\newcommand{\R}{\mathbb{R}} % reálná čísla

\renewcommand{\qed}{\hfill\blacksquare} % Quod Est Demonstratum (QED) 

% tohle je pro prostředí úkolů
\newenvironment{ukol}[2][]{\begin{trivlist} 
\item[\hskip \labelsep {\bfseries #1}\hskip \labelsep {\bfseries #2}]}{\end{trivlist}}

\linespread{1.15}
 
\begin{document}
 
% --------------------------------------------------------------
%                         Začni ZDE
% --------------------------------------------------------------
 
\title{ Evoluční algoritmy 1 \\ 2. domácí úkol
        } 
\author{Martin Gráf}
\date{30.10.2023}

\maketitle

Úkolem bylo ozkoušet hrstku genetických operátorů a měnit parametry pro prostý genetický algoritmus řešící Set partition problem.

\begin{ukol}{Jednoduché}

Počet ozkoušených variací se vymknul kontrole poměrně rychle. Oproti základu jsme zkoušeli turnajovou selekci, n-bodový crossover (Ne pro běh na bonus body),
fitness jako průměrný rozdíl, nebo fitness jako součet rozdílů všech přihrádek od všech ostatních. Dále jsme experimentovali s počtem generací, jedinců, 
a pravděpodobnostmi. Prakticky všechny ze zmíněných pokusů zlepšily výkon, ne všechny jsme ale z časových důvodů zaznamenali.

První graf znázorňuje turnajovou selekci, druhý turnajovou selekci s fitness jako průměrným rozdílem vah v jedinci, třetí nejlepší dosažený výsledek.

Nejlepších výsledků jsme dosáhli použitím turnajové selekce, fitness jako suma všech váhových rozdílů (Každá přihrádka s každou), pravděpodobností mutace 0.2,
pravděpodobností prohození přihrádky v rámci mutace 0.015, crossover pravděpodobností 0.85, 8500 jedinců, 1700 generací, a 20-jedincovým elitismem. Nejmenší 
nalezený rozdíl byl 12.


\begin{center}
	\begin{tabular}{ c c c }
		\includegraphics[width=.33\linewidth]{./tournament} & \includegraphics[width=.33\linewidth]{./average} & \includegraphics[width=.33\linewidth]{./default}\\ 
	\end{tabular}
\end{center}
\includegraphics[width=.7\linewidth]{./all}

\end{ukol}

\begin{ukol}{Informovaná mutace}

	Pokusili jsme se implementovat 2 různé informované mutace. První, která se neprokázala jako efektivní s danou pravděpodobností vybere nejtěžší a nejlehčí 
	(Nebo podle pravděpodbnosti vybere náhodné přihrádky) a z první vybrané přihrádky přesune nejlehčí objekt do nejtěžší. Nevůle přesunout těžší předměty
	se prokázala jako horší než náhoda. Druhá informovaná mutace se před náhodnou změnou bitu podívá, zdali se tím fitness jedince zlepšila. Pokud ano, mutaci 
	nechá, jinak ji zahodí. Dále jsme zkoušeli variantu tohoto přístupu kde jsme zahazovali i nezlepšující křížení. Kupodivu nakonec žádná z informovaných
	mutací nepřerčila neinformované. To ale může být i tím, že pro neinformované operátory jsme dlouho nalézali co nejlepší parametry, kdežto informované
	jsme obdobně rigorózně netestovali.
	\includegraphics[width=.7\linewidth]{./informed}
\end{ukol}

\end{document}
