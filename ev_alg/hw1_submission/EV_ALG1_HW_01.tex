\documentclass[a4paper]{article}
 
\usepackage[margin=1in]{geometry}
\usepackage{amsmath,amsthm,amssymb,bbm,wasysym}
\usepackage[czech]{babel}
\usepackage[utf8]{inputenc}
\usepackage[T1]{fontenc}
\usepackage{tikz}
\usepackage{graphicx}
\usepackage{enumitem}
\usepackage{tabto}
\usepackage{amsmath}

\graphicspath{ {./} }

\DeclareMathOperator{\Ex}{\mathbb{E}} % střední hodnota X pomocí $\Ex X$

\newcommand{\N}{\mathbb{N}} % přirozená čísla
\newcommand{\Z}{\mathbb{Z}} % celá čísla
\newcommand{\R}{\mathbb{R}} % reálná čísla

\renewcommand{\qed}{\hfill\blacksquare} % Quod Est Demonstratum (QED) 

% tohle je pro prostředí úkolů
\newenvironment{ukol}[2][]{\begin{trivlist} 
\item[\hskip \labelsep {\bfseries #1}\hskip \labelsep {\bfseries #2}]}{\end{trivlist}}

\linespread{1.15}
 
\begin{document}
 
% --------------------------------------------------------------
%                         Začni ZDE
% --------------------------------------------------------------
 
\title{ Evoluční algoritmy 1 \\ 1. domácí úkol
        } 
\author{Martin Gráf}
\date{16.10.2023}

\maketitle

Úkolem bylo ozkoušet hrstku genetických operátorů a měnit parametry pro prostý genetický algoritmus, OneMax problém, a problém střídajících se nul a jedniček v jedinci.

\begin{ukol}{OneMAX}

Pro OneMAX problém jsme zkrátka ozkoušeli algoritmus ze cvičení a záměnu za turnajovou selekci, která výrazně urychlila konvergenci.


	\begin{center}
		\begin{tabular}{ c c }
		 \includegraphics[width=.5\linewidth]{./default} & \includegraphics[width=.5\linewidth]{./tournament} \\ 
		\end{tabular}
	\end{center}

\end{ukol}

\begin{ukol}{Střídavé hodnoty}

Pro modré grafy jsme využili parametry ze cvičení, konkrétně tedy :

\begin{itemize}
	\item POP\_SIZE = 100
	\item IND\_LEN = 25
	\item CX\_PROB = 0.8
	\item MUT\_PROB = 0.05
	\item MUT\_FLIP\_PROB = 0.1
	\item N = 1
\end{itemize}

Červené grafy pak mají parametry:

\begin{itemize}
	\item POP\_SIZE = 100
	\item IND\_LEN = 25
	\item CX\_PROB = 0.7
	\item MUT\_PROB = 0.1
	\item MUT\_FLIP\_PROB = 0.1
	\item N = 2
\end{itemize}

Toto nejsou jediné ozkoušené parametry, zejména jsme experimentovali s každým parametrem zvlášť oproti referenčním hodnotám ze cvičení. Obecně snižování pravděpodobnosti křížení, zvyšování mutační pravděpodobnosti, nebo navyšování N pro křížení destabilizuje algoritmus a vykazuje tak déletrvající variabilitu hodnot. Mimo genetické operátory ze cvičení jsme přidali turnajovou selekci, křížení prohazující N částí jedince, nebo mutaci prohazující dvě části jednoho jedince.


\begin{center}
	\begin{tabular}{ c c }
	\includegraphics[width=.5\linewidth]{./AltDefault} & \includegraphics[width=.5\linewidth]{./ADefault} \\ 
	\includegraphics[width=.5\linewidth]{./AltS} & \includegraphics[width=.5\linewidth]{./AS} \\ 
	\includegraphics[width=.5\linewidth]{./AltN} & \includegraphics[width=.5\linewidth]{./AN} \\ 
	\includegraphics[width=.5\linewidth]{./AltTNS} & \includegraphics[width=.5\linewidth]{./ATNS} \\ 
	\end{tabular}
\end{center}


\end{ukol}

\end{document}
